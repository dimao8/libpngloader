\documentclass[a4paper,12pt]{article}
\usepackage[T2A]{fontenc}
\usepackage[utf8]{inputenc}
\usepackage[russian]{babel}
\usepackage{indentfirst}
\usepackage{hyperref}

\title{\Large{PNG Loader} \\ Руководство программиста \\ \textit{Ревизия 1.0}}
\date{2020}
\author{Хрущев Дмитрий aka DimaO}

\begin{document}
\maketitle
\newpage
\tableofcontents
\newpage
\section{Общее описание}
\label{section:common}
Библиотека \emph{pngloader} предназначена для загрузки PNG изображений из файла в буфер с сохранением PNG заголовка. От библиотеки libPNG отличается легковесностью и адаптацией под использование с приложениями OpenGL.

Основные особенности:
\begin{itemize}
\item[--] малый объем занимаемой памяти около 35 кБайт,
\item[--] более простой функционал, исключающий некоторые функции, например, сохранение в PNG файл‚ (в отличие от других библиотек, например libPNG),
\item[--] возможность загрузить PNG файл с последней строки (полезно при загрузке в качестве текстуры PNG),
\item[--] простой интерфейс (для загрузки требуется вызов всего одной функции),
\item[--] открытая лицензия, не исключающая использование в коммерческих проектах (LGPLv3).
\end{itemize}

Основные ограничения:
\begin{itemize}
\item[--] текущая версия (0.1) поддерживает глубину только 8 бит на канал,
\item[--] не поддерживает изображения, содержащие палитру,
\item[--] не поддерживает необязательные чанки PNG файла,
\item[--] декодирование основано на сторонней библиотеке (zlib).
\end{itemize}

\section{Использование библиотеки}
Для того, чтобы использовать библиотеку PNG loader, необходимо включить в исходные текст заголовок \texttt{pngloadr.h}

\leftskip=1.5cm
\texttt{\#include <pngloadr.h>}

\leftskip=0cm
Для успешной компоновки необходимо добавить в список компонуемых библиотек \texttt{-lpngloader} для статической компоновки или \texttt{-lpngloader.dll} для динамической компоновки. Если исользуется версия, скомпонованная динамически, потребуется добавить в список библиотев zlib \texttt{-lz}.

\section{Функции и типы библиотеки}

\subsection{Типы}

Тип \texttt{png\_error\_t} описывает тип возвращаемой PNGLoader ошибки. Возможные значения указаны в таблице \ref{table:png_error_t}.

\begin{table}[!h]
  \caption{Значения типа \texttt{png\_error\_t}}
  \label{table:png_error_t}
  \begin{tabular}{|c|p{8cm}|}
    \hline
    \multicolumn{1}{|c|}{\textbf{Обозначение}} & \multicolumn{1}{|c|}{\textbf{Описание}} \\ \hline
    \texttt{PNG\_NO\_ERROR} & Это значение всегда равно 0 и обозначает успешное завершение функции. \\ \hline
    \texttt{PNG\_ERROR\_NO\_SUCH\_FILE} & Данная ошибка указывает, что доступ к файлу невозможен. Это может происходить в случае отсутствия файла, его блокировки другим процессом или какими-либо прочими проблемами файлового ввода-вывода. \\ \hline
    \texttt{PNG\_ERROR\_NOT\_PNG} & Данная ошибка возвращается в случае, если модуль посчитал загружаемый файл не соответствующим спецификации PNG. Например, в файле отсутствует PNG сигнатура. \\ \hline
    \texttt{PNG\_ERROR\_WRONG\_CHUNK\_ORDER} & Редко возникающая ошибка, сигнализирующая, что чанк IDAT следует в PNG файле раньше, чем чанк IHDR. Эта ситуация возникает при неправильном формировании PNG файла и не позволяет определить истинный размер изображения. \\ \hline
    \texttt{PNG\_ERROR\_MALLOC} & Данная ошибка указывает на проблемы с выделением динамической памяти. Как правило, данная ошибка может возникнуть, если в заголовке PNG файла неверно указан размер изображения, или перед определенным чанком неверно указан его размер. \\ \hline
    \texttt{PNG\_ERROR\_EMPTY\_IMAGE} & Эта ошибка обозначает, что содержимое чанков IDAT отсутствет, или отсутствуют сами чанки IDAT (что допускается спецификацией PNG, но не имеет смысла). \\ \hline
    \texttt{PNG\_ERROR\_NOT\_SUPPORTED} & Данная ошибка указывает, что загружаемый формат файла \textbf{пока} не поддерживается текущей версией PNGLoader. О поддержке форматов см. \ref{section:common}. \\ \hline
    \texttt{PNG\_ERROR\_UNCOMPRESS} & Данная ошибка свидетельствует о проблемах с декомпрессией deflate-потока после его извлечения из PNG файла. Как правило, такая ошибка связана с проблемами с целостностью PNG файла. \\ \hline
  \end{tabular}
\end{table}

Тип \texttt{png\_header\_t} описывает заголов PNG файла. Его поля указаны в таблице \ref{table:png_header_t}.

\begin{table}[!h]
  \caption{Поля типа \texttt{png\_header\_t}}
  \label{table:png_header_t}
  \begin{tabular}{|c|c|p{8cm}|}
    \hline
    \multicolumn{1}{|c|}{\textbf{Обозначение}} & \multicolumn{1}{|c|}{\textbf{Тип}} & \multicolumn{1}{|c|}{\textbf{Описание}} \\ \hline
    \texttt{width} & \texttt{uint32\_t} & Ширина изображения в пикселях. \\ \hline
    \texttt{height} & \texttt{uint32\_t} & Высота изображения в пикселях. \\ \hline
    \texttt{depth} & \texttt{uint8\_t} & Глубина каждого канала PNG изображения. \\ \hline
    \texttt{color\_type} & \texttt{uint8\_t} & Тип изображения согласно спецификации на PNG файл. \\ \hline
    \texttt{compression} & \texttt{uint8\_t} & Тип сжатия согласно спецификации на PNG файл. \\ \hline
    \texttt{filter} & \texttt{uint8\_t} & Тип фильтрации изображения согласно спецификации на PNG файл. \\ \hline
    \texttt{interlace} & \texttt{uint8\_t} & Применение алгоритма чересстрочного сканирования согласно спецификации на PNG файл. \\ \hline
  \end{tabular}
\end{table}

\subsection{Функции}
Функция

\leftskip=1.5cm
\texttt{png\_error\_t LoadPNG(const char * file\_name, png\_header\_t * header, uint8\_t ** data, bool flip);}

\leftskip=0cm
Функция \texttt{LoadPNG} загружает PNG изображение из файла \texttt{file\_name} в область памяти, на которую указывает \texttt{data}. При этом в переменную \texttt{header} сохраняется оригинальное содержимое чанка IHDR, содержащего информацию о данных PNG. При установке переменной \texttt{flip} в значение \texttt{true}, изображение читается снизу вверх. Результат в формате \texttt{png\_error\_t} содержит код ошибки при загрузке.

Переменная \texttt{data} -- это указатель на указатель. Память выделяется автоматически, если загрузка происходит успешно. При неудаче значение указателя не определено. Безопасно освободить память можно только при помощи функции \texttt{FreePNG}.

Указатель на заголовок PNG типа \texttt{png\_header\_t} должен указывать на существующий заголовок, сожержимое которого будет переписано в результате выполенния функции \texttt{LoadPNG}. В случае неудачи содержимое этого заголовка не определено.

Функция

\leftskip=1.5cm
\texttt{void FreePNG(uint8\_t ** data);}

\leftskip=0cm
Функция \texttt{FreePNG} предназначена для безопасного освобождения памяти после выполнения функции \texttt{LoadPNG}. Ее необходимо вызывать с аргументом \texttt{data} после успешного выполнения функции \texttt{LoadPNG}, если данные изображения больше не нужны.

Функция

\leftskip=1.5cm
\texttt{void PNGLoaderVersion(char * str, size\_t n);}

\leftskip=0cm
Функция \texttt{PNGLoaderVersion} позволяет получить версию модуля в формате N.N.N.N, где N -- это числовое значение компонента версии. Версия копируется в строковый буфер \texttt{str} с нулевым символом на конце. Аргумент \texttt{n} должен сдержать размер буфера, чтобы не возникло выхода за его пределы. Для большинства версий размера буфера в 11 символов достаточно для хранения строки с версией.

\end{document}
